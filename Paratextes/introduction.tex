La notation musicale peut être distinguée en deux approches : un approche prescriptive et une approche descriptive \cite{battier2015}.
La notation prescriptive a pour but de décrire \og comment la musique doit sonner \fg.
Dans cette optique, la partition fait office de référence ou du moins de repère pour l'interprétation d'une pièce. 
La notation descriptive tente de retranscrire \og comment la musique a sonné \fg.
Ainsi, à des fins d'analyse, une pièce peut être caractérisée et fixée sur une partition.
Cependant, la partition reste l'outil privilégié du compositeur pour la communication de sa musique, et, au-delà de l'objet fini, constitue également un espace de travail.\\
Aussi, la préparation d'une pièce contemporaine par un ensemble musical se fait souvent en collaboration avec le compositeur. La partition constitue alors le support de la discussion et se voit même être modifiée pour les besoins de l'exécution de la pièce.\\
Comme le dit Carmine E. Cella, chercheur et compositeur à l'IRCAM, en parlant de l'écriture musicale : \og le créateur doit s'efforcer de trouver un compromis notationnel entre sa pensée et la réalisation pratique de sa pièce \fg. L'annexe \ref{sec:refletsDeLOmbre} donne deux exemples de partitions représentant la même partie de la pièce \textit{Reflets de l'ombre} (C. E. Cella, 2013), montrant l'adaptation de la notation à des fins d'exécution.
En définitive, une troisième approche de la notation musicale pourrait se profiler, celle d'une conception \textit{évolutive} de la partition.
\clearpage