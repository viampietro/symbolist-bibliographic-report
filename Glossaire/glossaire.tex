\newglossaryentry{neume}
{
        name = neume,
        description = {Symbole placé au-dessus des mots d'une phrase, donnant des indications mélodiques approximatives sur leur prononciation, qu'elle soit parlée ou chantée}        
}
\newglossaryentry{portee}
{
	name = {port{é}e},
	description = {Système de lignes horizontales, parallèles et équidistantes, sur lesquelles ou entre lesquelles sont placées les notes. (Ces lignes déterminent la hauteur des notes dans l'échelle des sons. On peut leur adjoindre des lignes supplémentaires et fragmentaires au-dessus ou en dessous pour étendre l'ambitus de la portée.)}	
}
\newglossaryentry{polyphonie}
{
	name = {polyphonie},
	description = {Combinaison de plusieurs voix ou parties mélodiques, dans une composition musicale}	
}
\newglossaryentry{tonalite}
{
	name = {tonalit{é}},
	description = {Une tonalité se définit comme une gamme de sept notes, désignée par sa tonique (appartenant à l'échelle diatonique) et son mode (majeur ou mineur)}
}