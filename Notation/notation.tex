\todo{Ajouter une intro pour la partie}
\section{Un peu d'histoire}
Comme le dit Jean-Yves Bosseur dans \cite{bosseur2005}, l'évolution perpétuelle de la notation musicale au fil du temps témoigne du caractère versatile de la pensée compositionnelle, qui est fonction de trois entités : le compositeur, l'interprète et le cadre socio-technologique d'une époque.
A travers l'histoire, la présente partie veut suggérer au lecteur la difficulté de l'écriture musicale, et la nécessité de sa réinvention constante.
Pour plus de clarté, l'histoire de la notation musicale est découpée en périodes historiques ne rendant pas entièrement compte du caractère continuel de l'évolution de la Musique.   


\paragraph{Antiquité et Moyen-Age} Le premier exemple d'une notation musicale apparaît au IIIe siècle av. J-C en Grèce avec la notation \textit{boécienne}. A cette époque, les notes de la gamme sont représentées par des lettres de l'alphabet.
Cette représentation est conçue dans une acception théorique pure visant à modéliser la relation entre les sons musicaux et les matières scientifiques, notamment les mathématiques.

Au IXe siècle, plusieurs exemples de parchemins retranscrivant des chants liturgiques sont annotés par des symboles appelés \glspl{neume} (voir annexe \ref{sec:exempleTexteNeume} page~\pageref{sec:exempleTexteNeume}). 

Ces symboles ont pour de but de décrire approximativement les inflexions mélodiques de la voix. Par exemple, le neume "point" (\textit{punctum}) indique le chant d'une note isolée; le neume "virgule" (\textit{virga}), représenté par un accent aigu, indique le chant à un intervalle supérieur vis à vis d'un neume "point".

Déjà, la notation neumatique témoigne de la volonté de fixation des caractéristiques du son chanté par le symbole. Ici, la dimension fixée est la hauteur relative des sons entre eux.

Par la suite, la notation carrée (1175-1225, voir annexe \ref{sec:exempleNotationCarree} page~\pageref{sec:exempleNotationCarree}), suivie par la notation noire (1250-1450), factorisent la profusion symbolique des neumes et coïncident avec la volonté de noter les pièces polyphoniques qui font leurs apparitions entre le Xème et XIème siècle.

La \gls{polyphonie}\footnote{Combinaison de plusieurs voix ou parties mélodiques, dans une composition musicale.} cristallise l'intérêt des compositeurs pour les relations de durée entre les différentes voix; petit à petit une notation rythmique va voir le jour.

Entre le XIIIème et le XIVème siècle, la barre verticale représentant la mesure fait son apparition sur la portée. La mesure regroupe, en ce temps, les ensembles de notes selon le mode rythmique employé dans la pièce musicale.

Dans la continuité de cette période, les symboles de notes se multiplieront et mèneront à une fixation absolue des valeurs rythmiques (c'est à dire, à un symbole est associé une durée).

Au XIIIème siècle, une \gls{portee} stable (quatre barres pour la liturgie, cinq barres pour les autres chants) fait son apparition. Ainsi, la hauteur des notes devient fixée strictement, notamment par l'apposition de clés (clé de Fa, de Ut et de Sol) au début de la partition.

Il est a noté que l'apparition des noms des notes de la gamme date du XIème siècle environ. Les noms correspondent au premières lettres d'un hymne à Saint Jean-Baptiste :

\begin{center}
	\textbf{UT} queant laxis - \textbf{RE}sonare fibris - \textbf{MI}ragestorum - \textbf{FA}muli tuorum -  \textbf{SOL}ve pollueti - \textbf{LA}billi reatum - \textbf{S}ancte \textbf{I}ohannes
\end{center}

La fin du XIVème siècle voit poindre "une phase de complication et d'intrications inégalées"\cite{bosseur2005} en termes de notation musicale. Vient s'ajouter à la variété des symboles musicaux un système de couleur qui adjoint une sémantique rythmique supplémentaire aux pièces musicales.

A cette époque, la partition perd sa fonctionnalité de transmettrice d'informations pour l'exécution musicale d'une pièce. Cette période correspond à l'âge d'or des moines-copistes, ce qui explique l'émulation notationnelle qui fait ressembler les partitions à de véritables œÒuvres graphiques, délaissant la compréhension des pièces par les interprètes.

Comme le fait remarquer Jean-Yves Bosseur, cette tendance se retrouvera au XXème siècle avec la diversité des notations apportée par la musique contemporaine. 
   
\paragraph{De la renaissance au romantisme} Au XVème siècle, l'invention de l'imprimerie conduit à la standardisation de l'écriture musicale. Les notes prennent peu à peu une forme arrondie du fait des techniques d'impression. Les fractions, indiquant la signature rythmique au sein des mesures, font également leurs apparitions. 

Malgré l'unification de la notation, une grande liberté d'interprétation est laissée aux musiciens exécutants. De la renaissance (du \textasciitilde XVème au XVIème siècle) jusqu'à l'époque baroque (du XVIIème au milieu du XVIIIème siècle), les partitions deviennent très épurées; elles ne conservent que la structure basique des pièces. Le reste est laissé à la discrétion de l'interprète et c'est, en cela, une manière de rester lié avec la tradition orale héritée du Moyen-Age.

\todo{Ajouter une image pour illustrée la notation épurée de la musique baroque en annexe}

La fin du XVIIème siècle voit apparaître de nouveaux symboles fixant les effets d'ornementations mélodiques sur la portée. Une moindre place est accordée à la liberté de jeu de l'interprète, qui devient de plus en plus dépendant de la notation musicale.

Progressivement, le métier d'interprète et de compositeur se scinde, et, la fin du XVIIIème siècle portant avec elle les idéaux de la révolution française, la singularité des oeuvres musicales est de plus en plus fixée sur la partition, dans la proclamation du droit moral inhérent de l'auteur sur ses oeuvres.

A l'ère du romantisme (\textasciitilde XIXème siècle), la notation s'intéresse à fixer plus finement les caractéristiques du son et de l'interprétation. Des nouveaux symboles apparaissent pour transcrire le doigté des instruments, la dynamique du son (\textit{nuances}) ou le mode d'accentuation des notes.

\todo{Ajouter une image pour illustrée l'augmentation des symboles servant à noter le son et l'interprétation en annexe}

\paragraph{Le XXème siècle} Par la suite, deux tendances se dessinent qui cohabiteront tout au long du XXème siècle. 

La première est à l'augmentation de la partition traditionnelle, dans le but de traduire avec précision les nouvelles formes musicales qui apparaissent à cette époque.

Par exemple, le compositeur Schoenberg introduit le \textit{dodécaphonisme} pour rompre avec la \gls{tonalite}\footnote{Une tonalité se définit comme une gamme de sept notes, désignée par sa tonique et son mode (majeur ou mineur). Par exemple, tonalité de Do majeur (Do = tonique, majeur = mode).}. Il imagine alors un nouveau type de portée qui mettrait à niveau égal chacun des douze sons de la gamme chromatique. Il voit dans l'augmentation de l'espace de représentation une manière de transcrire plus fidèlement sa musique.

Dans une autre démarche, le compositeur et ethnomusicologue Béla Bart\'{o}k essaye de retranscrire les musiques folkloriques à tradition orale entendues au cours de ses voyages d'investigation.
Cependant, il se heurte à la difficulté de noter avec un système limité des pratiques musicales particulières :

\begin{displayquote}[{\cite[94]{bosseur2005}}]
\og Dans les mélodies populaires, il y a beaucoup de sons étrangers, certains glissements de voix, des sons dont la hauteur ne peut être exactement précisée.\fg 
\end{displayquote}

\todo{Ajouter une image illustrant les symboles utilisés par Bela Bartok en annexe}

L'impuissance de Belà Bart\'{o}k à pouvoir noter certaines musiques lui fera dire que "la vraie partition se trouve sur les pistes du disque", résultat des enregistrements fait sur place.

Cette déclaration montre la limitation de la notation musicale, qui ne peut porter à elle seule toute la diversité et la complexité de la musique de cette époque.

Partant de ce constat, une autre tendance notationnelle apparaît, laissant plus de place à l'interprétation et voir à l'improvisation. 

En effet, les compositeurs de la deuxième moitié du XXème siècle (Stockhausen, Cage…) \og estiment volontiers que la notation, loin de s'efforcer de "conserver" les caractéristiques d'une œuvre -- tâche dont peuvent aujourd'hui se charger les moyens de reproduction mécanique avec une minutie inégalée par tout autre système de transcription -- devrait plutôt constituer un catalyseur pour le jeu musical. \fg (\cite[115]{bosseur2005}).

Ainsi, les partitions prennent un envers de plus en plus graphique, déconstruisant la notation "traditionnelle", et laissant le rôle de témoin d'une œuvre aux moyens technologiques.

La partition devient même une œuvre graphique en soi, et donne lieu à des expositions. La première prend place en 1959 à Donaueschingen, où sont présentées les partitions du compositeur Anestis Logothetis (voir annexe).
\todo[inline]{Ajouter une partition graphique de Anestis Logothetis en annexe}

Durant les années 50, la musique électroacoustique naît grâce aux outils électroniques de synthèse audio.
La composition se fait alors à même la matière sonore, sans intervention préalable d'une transcription symbolique des pièces.
Ainsi, la notation des œuvres électroacoustiques intervient souvent, dans une optique d'analyse, après leur production (\textit{représentation descriptive}, voir section \todo{ref vers section}).

De même, la notation de telles pièces s'écartent du symbolisme de la notation traditionnelle; une relation continue au temps est préférée au système métrique\footnote{Système basé sur la décomposition du temps selon une nombre de pulsations par minute.}.

La section \todo{ref vers section} reprend plus en détails les problématiques liées à la représentation de la musique électroacoustique. 

\bigskip

L'Histoire nous apprend que noter la Musique est une démarche en fluctuation constante. L'acte notationnel est fonction avant tout de la place que prennent le compositeur, l'interprète et même l'auditeur dans la tendance musicale liée à une époque.
De même, la notation se standardise aux époques où la portée constitue le médium d'échange par excellence et le support de l'interprétation : à la renaissance après l'invention de l'imprimerie, jusqu'à l'époque romantique.
En revanche, la notation se diversifie aux époques où la mémorisation des œuvres musicales est déléguée à d'autres médiums : au Moyen-Âge à la grande tradition orale, ou au XXème siècle avec les technologies d'enregistrement du son.   

Aujourd'hui, la diversité des approches et modes de production de la Musique rendre quasi impossible l'unification sa notation.
Au contraire, la direction inverse serait à privilégier :
\begin{displayquote}[{\cite[133]{bosseur2005}}]
\og Noter, ce n'est plus alors nécessairement indiquer une hauteur de son, un rythme…, noter, c'est aussi inventer une écriture.\fg 
\end{displayquote} 

