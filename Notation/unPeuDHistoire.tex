L'évolution perpétuelle de la notation musicale témoigne du caractère versatile de la pensée compositionnelle, qui est fonction de trois entités : le compositeur, l'interprète et le cadre socio-technologique d'une époque \cite{bosseur2005}.
 %A travers l'histoire, la présente section veut suggérer au lecteur la difficulté de l'écriture musicale, et la nécessité de sa réinvention constante. La présente section, qui ne saurait être exhaustive, est inspirée du livre \citetitle{bosseur2005} de Jean-Yves Bosseur. 
Inspirée du livre \citetitle{bosseur2005} de Jean-Yves Bosseur, la présente section % qui ne saurait être exhaustive, 
veut suggérer, à travers l'histoire, la difficulté de l'écriture musicale et la nécessité de sa réinvention constante.


\subsection{De l'antiquité au XXème siècle}
\label{subsec:antiquiteAuXX}
Le premier exemple d'une notation musicale apparaît au IIIe siècle av. J-C en Grèce avec la notation \textit{boécienne}. A cette époque, les notes de la gamme sont représentées par des lettres de l'alphabet.
Cette représentation abstraite met en parallèle les sons musicaux et les matières scientifiques, peu après que Pythagore ait théorisé mathématiquement les relations entre notes.
Au IXe siècle, plusieurs exemples de parchemins retranscrivant des chants liturgiques sont annotés par des symboles appelés \glspl{neume} (voir annexe \ref{sec:exempleTexteNeume}). 
Ces symboles ont pour but de décrire approximativement les inflexions mélodiques de la voix.
Déjà, la notation neumatique témoigne de la volonté de fixation des caractéristiques du son chanté par le symbole. Ici, la dimension fixée est la hauteur relative des sons entre eux.
Par la suite, la notation carrée (1175-1225, voir annexe \ref{sec:exempleNotationCarree}), suivie par la notation noire (1250-1450), factorisent la profusion symbolique des neumes et coïncident avec la volonté de noter les pièces polyphoniques qui font leurs apparitions entre le Xème et XIème siècle.
La \gls{polyphonie}\footnote{Combinaison de plusieurs voix ou parties mélodiques, dans une composition musicale.} cristallise l'intérêt des compositeurs pour les relations de durée entre les différentes voix; petit à petit une notation rythmique va voir le jour.
Entre le XIIIème et le XIVème siècle, la barre verticale représentant la mesure fait son apparition sur la portée. La mesure regroupe, en ce temps, les ensembles de notes selon le mode rythmique employé dans la pièce musicale.
Dans la continuité de cette période, les symboles de notes se multiplieront et mèneront à une fixation absolue des valeurs rythmiques (c'est à dire, à un symbole est associé une durée).
Au XIIIème siècle, une \gls{portee} stable (quatre barres pour la liturgie, cinq barres pour les autres chants) fait son apparition. Ainsi, la hauteur des notes devient fixée strictement, notamment par l'apposition de \glspl{clef} (clef de Fa, de Ut, et de Sol) au début de la partition.

La fin du XIVème siècle voit poindre \og une phase de complication et d'intrications inégalées \fg \cite[43]{bosseur2005} en termes de notation musicale. Vient s'ajouter à la variété des symboles musicaux un système de couleur qui adjoint une sémantique rythmique supplémentaire aux œuvres.
Cette période correspond à l'âge d'or des moines-copistes, ce qui explique l'émulation notationnelle qui fait ressembler les partitions à de véritables œuvres graphiques. Cette tendance se retrouvera au XXème siècle avec la diversité des notations apportée par la musique contemporaine \cite[44]{bosseur2005}. 
Au XVème siècle, l'invention de l'imprimerie conduit à la standardisation de l'écriture musicale. Les notes prennent peu à peu une forme arrondie du fait des techniques d'impression. Les fractions, indiquant la \gls{sigrythmique} au sein des mesures, font également leurs apparitions. 
Malgré l'unification de la notation de rigueur à cette époque, une grande liberté d'interprétation est laissée aux musiciens exécutants. De la renaissance (du XVème au XVIème siècle) jusqu'à l'époque baroque (du XVIIème au milieu du XVIIIème siècle), les partitions deviennent très épurées (voir annexe \ref{sec:exempleMusiqueBaroque}); elles ne conservent que la structure basique des pièces. Le reste est laissé à la discrétion de l'interprète : une manière de rester lié avec la tradition orale héritée du Moyen-Age.
La fin du XVIIème siècle voit apparaître de nouveaux symboles fixant les effets d'ornementations mélodiques sur la portée. Une moindre place est accordée à la liberté de jeu de l'interprète, qui devient de plus en plus dépendant de la partition.
A l'ère du romantisme (XIXème siècle), la notation s'attache à fixer plus finement les caractéristiques du son et de l'interprétation. De nouveaux symboles apparaissent pour transcrire le doigté des instruments (ou \textit{comment} produire les sons), la dynamique du son (\textit{nuances}) ou le mode d'accentuation des notes (voir annexe \ref{sec:exempleNotationInterpretation}).

Progressivement, le métier de compositeur s'affirme, et, la fin du XVIIIème siècle portant avec elle les idéaux de la révolution française, la singularité des œuvres musicales est de plus en plus fixée sur la partition, dans la proclamation du droit moral inhérent de l'auteur sur ses œuvres.

\subsection{Tendances notationnelles du XXème siècle à aujourd'hui}
\label{subsec:tendancesNotationnellesXX}
Deux tendances cohabitent tout au long du XXème siècle. 
La première est à l'augmentation de la partition traditionnelle, dans le but de traduire avec précision les nouvelles formes musicales qui apparaissent à cette époque.
Par exemple, le compositeur Schoenberg introduit le \textit{dodécaphonisme} pour rompre avec la \gls{tonalite}\footnote{Une tonalité se définit comme une gamme de sept notes, désignée par sa tonique et son mode (majeur ou mineur). Par exemple, tonalité de Do majeur (Do = tonique, majeur = mode).}. Il imagine alors un nouveau type de portée qui mettrait à niveau égal chacun des douze sons de la \gls{gammechromatique}. Il voit dans l'augmentation de l'espace de représentation une manière de transcrire plus fidèlement sa musique. Cependant, ce système ne sera adopté par aucun des musiciens de l'époque, y compris Schoenberg.
Conjointement, le compositeur et ethnomusicologue Béla Bart\'{o}k retranscrit les musiques folkloriques à tradition orale entendues au cours de ses voyages d'investigation. Pour ce faire, il invente de nouveaux symboles afin de fixer \og certains glissements de voix, des sons dont la hauteur ne peut être exactement précisée \fg \cite[94]{bosseur2005}.
Malgré ça, Belà Bart\'{o}k se trouve impuissant à pouvoir noter certaines musiques, ce qui lui fera dire que "la vraie partition se trouve sur les pistes du disque", résultat des enregistrements effectués sur place.
Cette déclaration montre la limitation de la notation musicale, qui ne peut porter à elle seule toute la diversité et la complexité de la musique de cette époque.

Partant de ce constat, une autre tendance notationnelle apparaît, laissant plus de place à l'interprétation et à l'improvisation. 
En effet, les compositeurs de la deuxième moitié du XXème siècle (Stockhausen, Cage, Brown…) \og estiment volontiers que la notation, loin de s'efforcer de "conserver" les caractéristiques d'une œuvre -- tâche dont peuvent aujourd'hui se charger les moyens de reproduction mécanique avec une minutie inégalée par tout autre système de transcription -- devrait plutôt constituer un catalyseur pour le jeu musical. \fg (\cite[115]{bosseur2005}).
Ainsi, les partitions prennent un envers de plus en plus graphique, déconstruisant la notation \og traditionnelle \fg, et laissant le rôle de témoin d'une œuvre aux moyens technologiques.
Des expositions de partitions sont même organisées. La première prend place en 1959 à Donaueschingen, où sont présentées les partitions du compositeur Anestis Logothetis (voir annexe \ref{sec:exempleAnestisLogothetis}).

Durant les années 50, la musique électroacoustique naît grâce aux outils informatiques de synthèse audiophonique. 
La composition se fait alors à même la matière sonore, sans transcription symbolique.
Ainsi, la notation des œuvres électroacoustiques intervient souvent, dans une optique d'analyse, après leur production (\textit{représentation descriptive}, voir l'introduction).
De même, la notation de telles pièces s'écartent du symbolisme de la notation traditionnelle. Par exemple, une représentation continue du temps y est généralement préférée au système métrique\footnote{Système basé sur la décomposition du temps selon une nombre de pulsations par minute.}.

\bigskip

L'Histoire nous apprend donc que noter la Musique est une démarche en fluctuation constante. L'acte notationnel est fonction avant tout de la place que prennent le compositeur, l'interprète et même l'auditeur dans la tendance musicale liée à une époque.
De même, la notation se standardise aux siècles où la portée constitue le médium d'échange par excellence et le support de l'interprétation : à la renaissance après l'invention de l'imprimerie, jusqu'à l'époque romantique.
En revanche, la notation se diversifie aux époques où la mémorisation des œuvres musicales est déléguée à d'autres médiums : au Moyen-Âge à la grande tradition orale, ou au XXème siècle avec les technologies d'enregistrement du son.
Aujourd'hui, la diversité des approches et modes de production de la Musique rend quasi impossible l'unification de sa notation.
Au contraire, la direction inverse serait à privilégier :
\begin{displayquote}[{\cite[133]{bosseur2005}}]
\og Noter, ce n'est plus alors nécessairement indiquer une hauteur de son, un rythme…, noter, c'est aussi inventer une écriture.\fg 
\end{displayquote}

\subsection{Problématiques liées à la notation de la musique contemporaine} 
\label{subsec:pbmatiquesMusiqueContemporaine}

%L'objectif du présent stage est de créer un outil informatique permettant de noter la musique contemporaine\footnote{La musique contemporaine représente les différents courants de musique savante apparus après la fin de la Seconde Guerre mondiale} et les compositions multimédias. Aussi, il est nécessaire de résumer les problématiques rencontrées lors de la notation de telles œuvres. 

La musique contemporaine fait souvent un usage inédit des instruments de musique en proposant de nouveaux modes de jeu. 
Aussi, de nouveaux symboles sont inventés par les compositeurs pour fixer ces effets sur la portée. 
%
L'un des premiers éléments, discuté précédemment, est donc le problème de la notation des gestes du musicien. 
Dans sa pièce \textit{In lieblicher Blaue…}, Karim Haddad représente par exemple sous forme de courbes reliant des symboles rythmiques la pression, le placement et la rapidité du jeu d'archer sur une contrebasse (voir annexe~\ref{fig:exempleKarimHaddad}).

Un deuxième problème fondamental est celui de la notation de l'\textit{électronique}\footnote{Le dénomination \textit{Live Electronics} est utilisée pour qualifier l'utilisation de dispositifs électroniques dans une pièce musicale.} dans les pièces contemporaines. 
Les compositeurs ont en effet besoin, par exemple, d'\textit{écrire} les effets appliqués aux sons (réverbération, amplification, écho…). 
Dans sa pièce \textit{Tutti 157} (1965), Karlheinz Stockhausen note avec des courbes les variations des fréquences de filtrage appliquées au son, ainsi que les mouvements du potentiomètre gérant l'amplification sonore (voir annexe \ref{sec:exempleStockhausen}). Stockhausen note également dans sa pièce la variation de la distance entre les micros et la source sonore (un gong). Cela montre que le système de captation, en plus du système de diffusion, peut être pris en compte lors de la notation de dispositifs électroniques.  
%
Noter la manière dont le son est diffusé et spatialisé est également un réel enjeu de la production musicale contemporaine \cite{ellberger2015}. 
Par exemple, dans sa pièce \textit{Tak-Sîm} (2012), Alireza Farhang modélise la spatialisation du son sur les quatre enceintes de diffusion par des trajectoires en pointillés sur la partition (voir annexe \ref{sec:exempleAlirezaFarhang}).

Les nouveaux symboles inventés par les compositeurs contemporains sont en général répertoriés et expliqués au début des partitions. Les installations techniques requises par les pièces y sont aussi documentées. De fait, la notation de la musique contemporaine est indissociable d'une forme de documentation accompagnant la lecture d'une partition. L'annexe \ref{sec:schemaInstallationFluoresceComplet} présente le schéma d'installation pour la pièce \textit{Fluoresce} (2012) de Rama Gottfried. Le schéma est présenté en début de partition.

Un dernier problème soulévé par les pratiques musicales actuelles vient de la cohabitation entre notation standard (traditionnelle) et symboles \og libres \fg dans les partitions de musique mixte\footnote{Type de musique qui mélange instruments traditionnels et informatique temps-réel.}. Le difficulté vient de l'alignement des symboles entre voix superposées. En effet, dans la notation standard, la relation entre la durée d'une note et l'espace horizontal occupé dans la portée n'est pas proportionnelle, pour des raisons esthétiques ou de lisibilité. Les logiciels de gravure musicale\footnote{Les logiciels dédiés à la production d'un rendu final (imprimable) des partitions.} implémentent, la plupart du temps, un algorithme d'optimisation pour calculer l'espacement horizontal entre les notes de la portée \cite{solomon2011}.
En revanche, beaucoup de symboles permettant de noter l'électronique détiennent une relation proportionnelle au temps (courbes, nuages de points, éléments graphiques arbitraires…).
Aussi, procéder à un alignement entre une voix instrumentale et une voix électronique est parfois délicat. Dans \cite{bresson2008}, Jean Bresson et Carlos Agon choisissent d'appliquer une distorsion graphique aux éléments qui observe une relation linéaire au temps. Ces éléments sont formatés pour être alignés avec les éléments rythmiques (les notes).

\bigskip
En prenant en compte l'ensemble des challenges soulevés ci-dessus, est-il encore possible, voir même utile, de vouloir noter la musique contemporaine ?
Même si la pluralité des approches musicales amène à l'éclatement d'une pratique notationnelle unique, la partition n'en reste pas moins \og un moyen pour le compositeur de penser sa musique, la décrire et la communiquer grâce à un système de représentation symbolique \fg \cite{bresson2008}.
En réponse à la nécessité d'une pluralité notationnelle, le projet \textit{symbolist}, qui constitue le cadre du présent stage de recherche, a été impulsé par Jean Bresson et Rama Gottfried, dans l'optique d'offrir aux compositeurs multimédias et de musique contemporaine un outil informatique pour la spécification de leur propre notation.
