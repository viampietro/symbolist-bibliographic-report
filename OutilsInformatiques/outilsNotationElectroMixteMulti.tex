Les outils présentés dans la section précédente ne font que porter sur informatique les pratiques notationnelles héritées de la gravure musicale sur papier en apportant, tout de même, toute la flexibilité de manipulation de la partition sous son format numérique.
La présente section propose une vue d'ensemble des logiciels de notation musicale se détachant de la notation traditionnelle pour privilégier une écriture des œuvres électroacoustiques et multimédias.
Notamment, la dimension interactive est fortement présente dans les partitions produites avec ces logiciels. Ces outils proposent un nouveau paradigme de représentation de la musique, en pensant la partition non plus uniquement  comme un support de fixation mais comme un \og contrôleur des paramètres d'une expérience sonore ou multimédia \fg \cite{gottfried2017}. 

\paragraph{Iannix} Le logiciel Iannix possède un statut particulier en ce qu'il propose à la fois une notation symbolique et \textit{morphologique} de la Musique \cite{coduys2003}. En effet, IanniX est une éditeur graphique en deux et trois dimensions possédant trois composantes principales pour l'écriture de partitions : la courbe, le curseur et le déclencheur. Le principe est le suivant : chaque courbe dessinée par l'utilisateur peut être associée un curseur qui va la parcourir à une vitesse paramétrée. Lorsqu'un curseur rencontrera un déclencheur (représenté par un cercle) sur sa route, ce même déclencheur lancera l'évènement qui lui était attaché.
Une infinité de courbes, curseurs et déclencheurs peuvent être intégrés à la partition, ce qui donne à IanniX un caractère multi-temporel, chaque curseur pouvant évoluer à une vitesse différente.
Par ailleurs, le logiciel est interfaçable avec une multitude d'autres programmes orientés "production sonore". Comme pour NoteAbility Pro, les déclencheurs de IanniX peuvent envoyer des messages via UDP à d'autres processus.

De plus, le dessin à base de courbes permet de représenter la forme de l'onde sonore ou encore l'évolution  des paramètres physiques du son en fonction du temps (évolution de l'intensité, la fréquence…). En ce sens, Iannix rend possible la notation morphologique de la Musique.
Également, tout autre symbole peut être composé à base de simples courbes, ouvrant les possibilités de création graphique à une hybridation entre symbolisme et morphologie.
L'annexe \ref{sec:exemplePartitionIannix} montre un exemple de partition graphique créée avec IanniX.

\paragraph{i-score} Le logiciel \textit{i-score} permet l'écriture de partitions \textit{interactives}. Une partition interactive organise temporellement les évènements composant une pièce musicale ou multimédia, et modélisent les contraintes régissant l'activation et la terminaison de ces évènements \cite{arias2017}.
La partition, interprétée par un ordinateur, réagit aux actions du performeur durant l'exécution de la pièce.     
\textit{i-score} propose une interface graphique et met à disposition de l'utilisateur un ensemble de symboles permettant de décrire la partition. Ces symboles sont divisés en plusieurs catégories : des symboles de contenu décrivant des objets musicaux; des symboles temporels décrivant des intervalles de temps et des évènements ponctuels; des symboles logiques décrivant des gardes conditionnelles, des déclencheurs et des boucles \cite{delahogue2016}.   
Une partition interactive créé avec \textit{i-score} prend donc la forme d'une machine à état régie par des contraintes temporelles (\textit{timed automaton}). Un exemple de partition interactive écrite avec i-score est montré en annexe \ref{sec:exempleIScore}.
 
L'interface \textit{i-score} peut être considérée comme une forme de notation des musiques éléctroacoustiques interactives \cite{assayag2008}. D'ailleurs, \textit{i-score} met l'accent sur la notation des processus informatiques, leurs relations et leurs actions. Cependant, le logiciel n'intègre pas la notation standard de la musique à son interface, rendant par exemple impossible la composition de musique mixte.   

\paragraph{\textsc{INScore}} \textsc{INScore} est un framework pour l'édition de partitions \textit{augmentées}, développé au Grame \cite{fober2012}. Il permet de représenter une partition à partir du \textit{GuidoEngine} (voir section \ref{subsec:notationABaseCompilateurs}), mais également d'y intégrer des images, des vidéos ou d'y décrire des formes d'ondes. \textsc{INScore} ajoute une grande interactivité aux partitions en rendant possible la programmation d'animations graphiques, la lecture de fichiers audio, et le déclenchement d'évènements.    
Le workflow du framework est le suivant : l'utilisateur écrit un script Inscore, qui est un langage de programmation inspiré du protocole OSC (voir section \ref{subsec:interoperabilite}), intégrant également un interpréteur Javascript; le script est transpilé en C++, puis le rendu graphique de la partition est effectué avec le framework \textit{Qt}.
La syntaxe des scripts \textsc{INScore} est basé sur la structure \lstinline[language=html]|<oscaddress> <method> <args>|. Voici un exemple, d'appel de méthode écrit avec \textsc{INScore} :
\begin{lstlisting}[language=html]
/ITL/scene/score background-color "green"
\end{lstlisting}  

L'adresse OSC \textit{/ITL/scene/score} pointe l'objet \textit{score}, la partition graphique présentée à l'utilisateur; \textit{background}\textit{-color} désigne la méthode à appeler sur l'objet pointé; la chaîne de caractères \textit{green} est l'argument passé à la méthode \textit{background-color}.
Sur cette même base, \textsc{INScore} propose un ensemble de méthodes pour la mise en forme de la partition et la construction d'interactivité \cite{fober2017}.
Par exemple, la méthode \textit{map} permet d'aligner les notes d'une portée avec une autre figure graphique en définissant des points d'encrage. La méthode \textit{map} est la réponse d'\textsc{INScore} au problème de superposition souligné en section \ref{subsec:pbmatiquesMusiqueContemporaine}.
La méthode \textit{sync} permet de lier deux éléments de la partition; un élément jouera le rôle de maître et l'autre d'esclave, asservi graphiquement à son maître. Ce mécanisme est utilisé pour définir un curseur lié à la partition, qui servira de tête de lecture une fois asservi.
La méthode \textit{watch} associe un écouteur d'évènement à un objet de la partition. Par exemple, un objet curseur peut écouter sa position sur l'axe temporel et déclencher une action lors de son entrée dans un intervalle donné (réaction à l'évènement \textit{timeEnter}). Ce mécanisme peut être utilisé pour simuler la lecture d'une partition.

Même si \textsc{INScore} propose de nombreuses méthodes pour l'intégration d'interactivité dans la partition, sa manipulation porte la lourdeur d'un langage de programmation à syntaxe textuelle. Comme pour les outils à base de compilateur (voir section \ref{subsec:notationABaseCompilateurs}), \textsc{INScore} ne prend pas en compte les utilisateurs non-informaticiens, majoritairement représentés chez les compositeurs.        
