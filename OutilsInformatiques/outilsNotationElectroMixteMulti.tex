\paragraph{Iannix} Le logiciel Iannix possède un statut particulier en ce qu'il propose à la fois une notation symbolique et \textit{morphologique} de la Musique. En effet, IanniX est une éditeur graphique en deux et trois dimensions possédant trois composantes principales pour l'écriture de partitions : la courbe, le curseur et le déclencheur. Le principe est le suivant, chaque courbe dessinée par l'utilisateur peut être associée un curseur qui va la parcourir à une vitesse paramétrée. Lorsqu'un curseur rencontrera un déclencheur (représenté par un cercle) sur sa route, ce même déclencheur lancera l'évènement qui lui était attaché.
Une infinité de courbes, curseurs et déclencheurs peuvent être intégrés à la partition, ce qui donne à IanniX un caractère multi-temporel, chaque curseur pouvant évoluer à une vitesse différente.
Par ailleurs, le logiciel est interfaçable avec une multitude d'autres programmes orientés "production sonore". Comme pour NoteAbility Pro, les déclencheurs de IanniX peuvent envoyer des messages via UDP à d'autres processus.

De plus, le dessin à base de courbes permet de représenter la forme de l'onde sonore ou encore l'évolution  des paramètres physiques du son en fonction du temps (évolution de l'intensité, la fréquence…). En ce sens, Iannix rend possible la notation morphologique de la Musique.
Également, tout autre symbole peut être composé à base de simples courbes, ouvrant les possibilités de création graphique à une hybridation entre symbolisme et morphologie.
L'annexe \ref{sec:exemplePartitionIannix} montre un exemple de partition graphique créée avec IanniX.

\paragraph{i-score} Le logiciel \textit{i-score} permet l'écriture de partitions \textit{interactives}. Une partition interactive organise temporellement les évènements composant une pièce musicale, et modélisent les contraintes régissant l'activation et la terminaison de ces évènements \cite{arias2017}.
La partition, interprétée par un ordinateur, réagit aux actions du performeur durant l'exécution de la pièce.     
La création de telles partitions fait écho à la pratique du \textit{live coding}, qui \og implique l'écriture et la modification de programmes informatiques générant de la musique en temps-réel \fg \cite{sorensen2009}.
\textit{i-score} propose une interface graphique et met à disposition de l'utilisateur un ensemble de symboles permettant de décrire la partition. Ces symboles sont divisés en plusieurs catégories : des symboles de contenu décrivant des objets musicaux; des symboles temporels décrivant des intervalles de temps et des évènements ponctuels; des symboles logiques décrivant des gardes conditionnelles, des déclencheurs et des boucles \cite{delahogue2016}.   
Une partition interactive créé avec \textit{i-score} prend donc la forme d'une machine à état régit par des contraintes temporelles (\textit{timed automaton}).Un exemple de partition interactive écrite avec i-score est montré en annexe \ref{sec:exempleIScore}.
 
Le logiciel i-score a été créé pour aider les compositeurs de musiques électroacoustiques à noter leurs pièces \cite{assayag2008}. D'ailleurs, \textit{i-score} met l'accent sur la notation des processus informatiques, leurs relations et leurs actions. Cependant, le logiciel n'intègre pas la notation standard de la musique à son interface, rendant par exemple impossible la composition de musique mixte, qui implique une interaction entre un ordinateur et un instrumentiste.   
