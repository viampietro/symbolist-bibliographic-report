\lhead{\textit{NOTATION MORPHOLOGIQUE}}
Comme vu en section \ref{subsec:pbmatiquesMusiqueContemporaine}, la notation morphologique de la Musique consiste à représenter par des courbes paramètre/temps les différentes caractéristiques physiques du son.
Au cours des années 50, la représentation morphologique de la Musique fait écho à l'utilisation des bandes magnétiques dans le processus de production; bandes magnétiques sur lesquelles étaient présentes les formes d'ondes décrivant l'enveloppe du son.\\
Clairement donc, la notation morphologique découle de l'utilisation de l'électronique "analogique" lors des premiers pas de la musique électroacoustique\cite{bosseur2005}.

Cependant, une telle écriture est loin de se suffire à elle-même. Dans \cite{gottfried2017}, Rama Gottfried souligne le manque de résonance de la notation morphologique pour l'oreille du compositeur. Aussi, les logiciels d'analyse de partitions, comme \textit{iAnalyse}, intègre souvent la notation morphologique en complément de la notation traditionnelle.

Néanmoins, l'écriture morphologique reste le moyen le plus naturel de noter l'intervention de l'électronique/informatique dans les pièces musicales. Aussi, des logiciels comme \textit{Ascograph} ou \textit{IanniX} mêlent-ils la diversité des notations pour offrir une vision complète de la partition.   

\subsection{Notation morphologique pour l'analyse de partitions}
\label{subsec:outilsMorphologiquesAnalyse}

Cette partie présente deux logiciels utilisant une approche morphologique de la notation à des fins d'analyse musicale.

\paragraph{iAnalyse} Le logiciel iAnalyse, créé par Pierre Couprie, se place dans le domaine de \og l'analyse musicale assistée par ordinateur \fg \cite{couprie2008}. iAnalyse permet d'annoter une portée traditionnelle via de nombreux outils de marquage graphique et textuel. Notamment, le logiciel prévoit l'adjonction d'un fichier audio à la portée; fichier audio pouvant lui-même être annoté, et dont la lecture peut être synchronisée avec le défilement de la portée. Dans l'interface graphique, la forme d'ondes extraite du fichier audio est disposée en-dessous de la partition, procurant une double vue, symbolique et morphologique, de la même pièce.\\
Analyser, c'est aussi chercher un nouvel angle pour la représentation d'un objet. De fait, l'écriture morphologique du son prend naturellement sa place dans le logiciel \textit{iAnalyse}. 

\paragraph{EAnalysis} \textit{EAnalysis}, développé au MTI à l'Université de De Monfort, Leicester, est le pendant de \textit{iAnalyse} pour la musique électroacoustique. A l'inverse d'\textit{iAnalyse}, \textit{EAnalysis} utilise les multiples représentations morphologiques du son comme base pour l'étude musicologique d'une pièce. Pour l'instant, l'interface graphique d'\textit{EAnalysis} propose deux vues, une vue en forme d'ondes, et un spectrogramme\footnote{Représentation de l'évolution d'un son dans un repère orthonormé. Le temps est représenté sur l'axe des abscisses et les fréquences composant le son sur l'axe des ordonnées. Chaque fréquence composant le son à un instant $t$ est associée à une couleur caractérisant son intensité sonore.} pour décrire l'œuvre analysée.
Chaque vue peut-être annotée par deux types de symboles appelés \textit{graphic events} et \textit{analytic events}. Les \textit{graphic events} représentent des symboles sans sémantique associée : carré, rond, ligne…
L'utilisateur peut adjoindre une sémantique particulière à ce type de symboles, en décrivant textuellement leur signification.\\
Les \textit{analytic events} sont des symboles à la sémantique connue, inventer par les différents systématiciens de la musique électroacoustique. Notamment, les symboles issus de la classification typo-morphologique des objets musicaux par Pierre Schaeffer sont intégrés en tant que \textit{analytic events} à \textit{EAnalysis}. Une vue de l'interface graphique de \textit{EAnalysis} est proposée en annexe \ref{sec:exempleVueEAnalysis}.

\paragraph{Les stations audionumériques : une forme de notation morphologique}

    