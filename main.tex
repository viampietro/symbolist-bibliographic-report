\documentclass[a4paper,oneside]{book}

%% Language and font encodings
\usepackage[frenchb]{babel}

\usepackage[utf8]{inputenc}
\usepackage[T1]{fontenc}

\usepackage{verbatim}
\usepackage{enumitem}

% creation de la commande circled, mot entoure d'un cercle
\newcommand*\circled[1]{\tikz[baseline=(char.base)]{% <---- BEWARE
            \node[shape=circle,draw,inner sep=2pt, color=pumpkin] (char) {#1};}}

% quotes package
\usepackage[autostyle, maxlevel = 2]{csquotes}

% bibliography package            
\usepackage[backend = biber, style = numeric]{biblatex}   
\addbibresource{references.bib}            

% glossary package
\usepackage[toc]{glossaries}

% loading glossary file
\loadglsentries{Glossaire/glossaire.tex} 

\makeglossaries

% package for references and links 
\usepackage{caption}

% package for insertion of pdf pages
\usepackage{pdfpages}

% pour les images
\usepackage{graphicx}
\usepackage{caption} 
\usepackage{float} 
\usepackage{wrapfig}

%% pour les tableaux
\usepackage{array}
\usepackage{tabularx}
\usepackage{multirow}
\usepackage{slashbox}
\usepackage{colortbl}
\usepackage{framed}
\usepackage{adjustbox}

% pour les maths
\usepackage{amsmath}
\usepackage{amsfonts}
\usepackage{amssymb}
\usepackage{mathrsfs}  
\usepackage{pifont}
\newcommand{\cmark}{\ding{51}}
\newcommand{\xmark}{\ding{55}}

%% package for landscape page view
\usepackage{lscape}

%% pour dessiner des graphiques et des schemas
\usepackage{tikz}
\usepackage{tkz-graph}
\usetikzlibrary{calc, decorations.pathreplacing}

% Pgfplots
\usepackage{pgfplotstable}
\usepackage{pgfplots}
\usetikzlibrary{pgfplots.groupplots}
\pgfplotsset{compat=1.12}

% définitions des couleurs
\usepackage{color}
  \definecolor{grey}{rgb}{0.4,0.4,0.4}
  \definecolor{blue}{rgb}{0.2,0.3,0.6}
  \definecolor{teal}{rgb}{0.1,0.4,0.4}
  \definecolor{green}{rgb}{0.1,0.7,0.2}
  \definecolor{red}{rgb}{0.8,0.1,0.2}
  \definecolor{pumpkin}{rgb}{0.9, 0.3, 0}

%% Pour l'intégration de code SQL
\usepackage{listings} 
\usepackage{listingsutf8}
\lstloadlanguages{JAVA, SQL}
\lstset{% affichage du code par défaut
    inputencoding=utf8/latin1,
    basicstyle=\footnotesize\sf,
    morecomment=[s]{/*}{*/},
    morecomment=[l]{//}, 
    keywordstyle=\sffamily\bfseries\color{teal},
    commentstyle=\itshape\color{grey},
    stringstyle=\rmfamily\color{pumpkin},
    tabsize=2, frame=single, breaklines=true,
    showspaces=false, showstringspaces=false,extendedchars=true, 
    numbers=left, numberstyle=\tiny,
    extendedchars=true,
    literate={\$}{{{\$}}}1 {é}{{\'e}}1,    
}

%% definition de style pour une ligne entiere d'un tableau
\newcolumntype{+}{>{\global\let\currentrowstyle\relax}}
\newcolumntype{^}{>{\currentrowstyle}}
\newcommand{\rowstyle}[1]{\gdef\currentrowstyle{#1}%
#1\ignorespaces
}

%% Sets page size and margins
\usepackage[a4paper,top=2cm,bottom=2cm,left=2cm,right=2cm,marginparwidth=1.75cm]{geometry}

%% Pour ecrire des algorithmes
\usepackage[vlined,ruled,linesnumbered]{algorithm2e}

%% Useful packages
\usepackage{amsmath}
\usepackage{amssymb}
\usepackage{amsthm}
\theoremstyle{definition}
\newtheorem{example}{Example}

\usepackage[colorinlistoftodos]{todonotes}
\usepackage[colorlinks=true, allcolors=blue]{hyperref}
\usepackage{nameref}

\usepackage{titlesec}
\titleformat{\chapter}[display]{\flushleft\Huge\itshape}{\quad}{0.5em}{}[]
\titleformat{\paragraph}[runin]{\normalfont\normalsize\bfseries}{}{0pt}{}

% leaves out chapter numbers in section numbering
\renewcommand*\thesection{\arabic{section}}

% command defining new chapter type, to be used whith roman page numbering
\newcommand\romanchapter[1]{
  \chapter*{#1}
  \markboth{\MakeUppercase{#1}}{}
  \addcontentsline{toc}{chapter}{#1}
}

\usepackage{fancyhdr}
\setlength{\headheight}{15.2pt}
\pagestyle{fancy}

\begin{document}

\begin{titlepage}
\begin{center}
\begin{sffamily}

{\large
Faculté de Sciences de Montpellier \\[.5cm]
Master 1 AIGLE\\2016 -- 2017\\[2cm]
}


% Title
\rule{\textwidth}{1.6pt}\vspace*{-\baselineskip}\vspace*{2pt} 
\rule{\textwidth}{0.4pt}\\[\baselineskip]
{\LARGE
Projet tuteuré\\[0.7\baselineskip]
\Huge
Notation Musicale\\[0.7\baselineskip]
\includegraphics[width=0.2\textwidth]{Paratextes/i/logo.jpg}
\\[0.5\baselineskip]
Rapport bibiliographique
}\\[0.2\baselineskip] 
\rule{\textwidth}{0.4pt}\vspace*{-\baselineskip}\vspace{3.2pt}
\rule{\textwidth}{1.6pt}\\[\baselineskip]
\vspace*{2\baselineskip}

% Author and supervisor
\noindent
\begin{center}
     \large
    \emph{\textbf{Étudiant:}}\\
    Vincent Iampietro \\
    \smallskip
    \large
    \emph{\textbf{Encadrant:}}\\
    Jean Bresson\\
\end{center}%



\end{sffamily}
\end{center}
\end{titlepage}

%%%%% PAGE DE GARDE %%%%%
\pagenumbering{Roman}

\setcounter{tocdepth}{1} % profondeur du sommaire, 1 pour sections uniquement

%%%%% SOMMAIRE %%%%%%
\tableofcontents

%%%%% INTRODUCTION %%%%%
\chapter{Introduction}
	La notation musicale peut être distinguée en deux approches : un approche prescriptive et une approche descriptive \cite{battier2015}.
La notation prescriptive a pour but de décrire \og comment la musique doit sonner \fg.
Dans cette optique, la partition fait office de référence ou du moins de repère pour l'interprétation d'une pièce. 
La notation descriptive tente de retranscrire \og comment la musique a sonné \fg.
Ainsi, à des fins d'analyse, une pièce peut être caractérisée et fixée sur une partition.
Cependant, la partition reste l'outil privilégié du compositeur pour la communication de sa musique, et, au-delà de l'objet fini, constitue également un espace de travail.\\
Aussi, la préparation d'une pièce contemporaine par un ensemble musical se fait souvent en collaboration avec le compositeur. La partition constitue alors le support de la discussion et se voit même être modifiée pour les besoins de l'exécution de la pièce.\\
Comme le dit Carmine E. Cella, chercheur et compositeur à l'IRCAM, en parlant de l'écriture musicale : \og le créateur doit s'efforcer de trouver un compromis notationnel entre sa pensée et la réalisation pratique de sa pièce \fg. L'annexe \ref{sec:refletsDeLOmbre} donne deux exemples de partitions représentant la même partie de la pièce \textit{Reflets de l'ombre} (C. E. Cella, 2013), montrant l'adaptation de la notation à des fins d'exécution.
En définitive, une troisième approche de la notation musicale pourrait se profiler, celle d'une conception \textit{évolutive} de la partition.
\clearpage

\stepcounter{page}
\pagenumbering{arabic}

%%%%% CHAPITRE "DE LA NOTATION MUSICALE" %%%%%
\chapter{De la notation musicale}
	\todo{Ajouter une intro pour le chapitre}	
	
	\section{Un peu d'histoire}
	\label{sec:unPeuDHistoire}
	L'évolution perpétuelle de la notation musicale témoigne du caractère versatile de la pensée compositionnelle, qui est fonction de trois entités : le compositeur, l'interprète et le cadre socio-technologique d'une époque \cite{bosseur2005}.
 %A travers l'histoire, la présente section veut suggérer au lecteur la difficulté de l'écriture musicale, et la nécessité de sa réinvention constante. La présente section, qui ne saurait être exhaustive, est inspirée du livre \citetitle{bosseur2005} de Jean-Yves Bosseur. 
Inspirée du livre \citetitle{bosseur2005} de Jean-Yves Bosseur, la présente section % qui ne saurait être exhaustive, 
veut suggérer, à travers l'histoire, la difficulté de l'écriture musicale et la nécessité de sa réinvention constante.


\subsection{De l'antiquité au XXème siècle}
\label{subsec:antiquiteAuXX}
Le premier exemple d'une notation musicale apparaît au IIIe siècle av. J-C en Grèce avec la notation \textit{boécienne}. A cette époque, les notes de la gamme sont représentées par des lettres de l'alphabet.
Cette représentation abstraite met en parallèle les sons musicaux et les matières scientifiques, peu après que Pythagore ait théorisé mathématiquement les relations entre notes.
Au IXe siècle, plusieurs exemples de parchemins retranscrivant des chants liturgiques sont annotés par des symboles appelés \glspl{neume} (voir annexe \ref{sec:exempleTexteNeume}). 
Ces symboles ont pour but de décrire approximativement les inflexions mélodiques de la voix.
Déjà, la notation neumatique témoigne de la volonté de fixation des caractéristiques du son chanté par le symbole. Ici, la dimension fixée est la hauteur relative des sons entre eux.
Par la suite, la notation carrée (1175-1225, voir annexe \ref{sec:exempleNotationCarree}), suivie par la notation noire (1250-1450), factorisent la profusion symbolique des neumes et coïncident avec la volonté de noter les pièces polyphoniques qui font leurs apparitions entre le Xème et XIème siècle.
La \gls{polyphonie}\footnote{Combinaison de plusieurs voix ou parties mélodiques, dans une composition musicale.} cristallise l'intérêt des compositeurs pour les relations de durée entre les différentes voix; petit à petit une notation rythmique va voir le jour.
Entre le XIIIème et le XIVème siècle, la barre verticale représentant la mesure fait son apparition sur la portée. La mesure regroupe, en ce temps, les ensembles de notes selon le mode rythmique employé dans la pièce musicale.
Dans la continuité de cette période, les symboles de notes se multiplieront et mèneront à une fixation absolue des valeurs rythmiques (c'est à dire, à un symbole est associé une durée).
Au XIIIème siècle, une \gls{portee} stable (quatre barres pour la liturgie, cinq barres pour les autres chants) fait son apparition. Ainsi, la hauteur des notes devient fixée strictement, notamment par l'apposition de \glspl{clef} (clef de Fa, de Ut, et de Sol) au début de la partition.

La fin du XIVème siècle voit poindre \og une phase de complication et d'intrications inégalées \fg \cite[43]{bosseur2005} en termes de notation musicale. Vient s'ajouter à la variété des symboles musicaux un système de couleur qui adjoint une sémantique rythmique supplémentaire aux œuvres.
Cette période correspond à l'âge d'or des moines-copistes, ce qui explique l'émulation notationnelle qui fait ressembler les partitions à de véritables œuvres graphiques. Cette tendance se retrouvera au XXème siècle avec la diversité des notations apportée par la musique contemporaine \cite[44]{bosseur2005}. 
Au XVème siècle, l'invention de l'imprimerie conduit à la standardisation de l'écriture musicale. Les notes prennent peu à peu une forme arrondie du fait des techniques d'impression. Les fractions, indiquant la \gls{sigrythmique} au sein des mesures, font également leurs apparitions. 
Malgré l'unification de la notation de rigueur à cette époque, une grande liberté d'interprétation est laissée aux musiciens exécutants. De la renaissance (du XVème au XVIème siècle) jusqu'à l'époque baroque (du XVIIème au milieu du XVIIIème siècle), les partitions deviennent très épurées (voir annexe \ref{sec:exempleMusiqueBaroque}); elles ne conservent que la structure basique des pièces. Le reste est laissé à la discrétion de l'interprète : une manière de rester lié avec la tradition orale héritée du Moyen-Age.
La fin du XVIIème siècle voit apparaître de nouveaux symboles fixant les effets d'ornementations mélodiques sur la portée. Une moindre place est accordée à la liberté de jeu de l'interprète, qui devient de plus en plus dépendant de la partition.
A l'ère du romantisme (XIXème siècle), la notation s'attache à fixer plus finement les caractéristiques du son et de l'interprétation. De nouveaux symboles apparaissent pour transcrire le doigté des instruments (ou \textit{comment} produire les sons), la dynamique du son (\textit{nuances}) ou le mode d'accentuation des notes (voir annexe \ref{sec:exempleNotationInterpretation}).

Progressivement, le métier de compositeur s'affirme, et, la fin du XVIIIème siècle portant avec elle les idéaux de la révolution française, la singularité des œuvres musicales est de plus en plus fixée sur la partition, dans la proclamation du droit moral inhérent de l'auteur sur ses œuvres.

\subsection{Tendances notationnelles du XXème siècle à aujourd'hui}
\label{subsec:tendancesNotationnellesXX}
Deux tendances cohabitent tout au long du XXème siècle. 
La première est à l'augmentation de la partition traditionnelle, dans le but de traduire avec précision les nouvelles formes musicales qui apparaissent à cette époque.
Par exemple, le compositeur Schoenberg introduit le \textit{dodécaphonisme} pour rompre avec la \gls{tonalite}\footnote{Une tonalité se définit comme une gamme de sept notes, désignée par sa tonique et son mode (majeur ou mineur). Par exemple, tonalité de Do majeur (Do = tonique, majeur = mode).}. Il imagine alors un nouveau type de portée qui mettrait à niveau égal chacun des douze sons de la \gls{gammechromatique}. Il voit dans l'augmentation de l'espace de représentation une manière de transcrire plus fidèlement sa musique. Cependant, ce système ne sera adopté par aucun des musiciens de l'époque, y compris Schoenberg.
Conjointement, le compositeur et ethnomusicologue Béla Bart\'{o}k retranscrit les musiques folkloriques à tradition orale entendues au cours de ses voyages d'investigation. Pour ce faire, il invente de nouveaux symboles afin de fixer \og certains glissements de voix, des sons dont la hauteur ne peut être exactement précisée \fg \cite[94]{bosseur2005}.
Malgré ça, Belà Bart\'{o}k se trouve impuissant à pouvoir noter certaines musiques, ce qui lui fera dire que "la vraie partition se trouve sur les pistes du disque", résultat des enregistrements effectués sur place.
Cette déclaration montre la limitation de la notation musicale, qui ne peut porter à elle seule toute la diversité et la complexité de la musique de cette époque.

Partant de ce constat, une autre tendance notationnelle apparaît, laissant plus de place à l'interprétation et à l'improvisation. 
En effet, les compositeurs de la deuxième moitié du XXème siècle (Stockhausen, Cage, Brown…) \og estiment volontiers que la notation, loin de s'efforcer de "conserver" les caractéristiques d'une œuvre -- tâche dont peuvent aujourd'hui se charger les moyens de reproduction mécanique avec une minutie inégalée par tout autre système de transcription -- devrait plutôt constituer un catalyseur pour le jeu musical. \fg (\cite[115]{bosseur2005}).
Ainsi, les partitions prennent un envers de plus en plus graphique, déconstruisant la notation \og traditionnelle \fg, et laissant le rôle de témoin d'une œuvre aux moyens technologiques.
Des expositions de partitions sont même organisées. La première prend place en 1959 à Donaueschingen, où sont présentées les partitions du compositeur Anestis Logothetis (voir annexe \ref{sec:exempleAnestisLogothetis}).

Durant les années 50, la musique électroacoustique naît grâce aux outils informatiques de synthèse audiophonique. 
La composition se fait alors à même la matière sonore, sans transcription symbolique.
Ainsi, la notation des œuvres électroacoustiques intervient souvent, dans une optique d'analyse, après leur production (\textit{représentation descriptive}, voir l'introduction).
De même, la notation de telles pièces s'écartent du symbolisme de la notation traditionnelle. Par exemple, une représentation continue du temps y est généralement préférée au système métrique\footnote{Système basé sur la décomposition du temps selon une nombre de pulsations par minute.}.

\bigskip

L'Histoire nous apprend donc que noter la Musique est une démarche en fluctuation constante. L'acte notationnel est fonction avant tout de la place que prennent le compositeur, l'interprète et même l'auditeur dans la tendance musicale liée à une époque.
De même, la notation se standardise aux siècles où la portée constitue le médium d'échange par excellence et le support de l'interprétation : à la renaissance après l'invention de l'imprimerie, jusqu'à l'époque romantique.
En revanche, la notation se diversifie aux époques où la mémorisation des œuvres musicales est déléguée à d'autres médiums : au Moyen-Âge à la grande tradition orale, ou au XXème siècle avec les technologies d'enregistrement du son.
Aujourd'hui, la diversité des approches et modes de production de la Musique rend quasi impossible l'unification de sa notation.
Au contraire, la direction inverse serait à privilégier :
\begin{displayquote}[{\cite[133]{bosseur2005}}]
\og Noter, ce n'est plus alors nécessairement indiquer une hauteur de son, un rythme…, noter, c'est aussi inventer une écriture.\fg 
\end{displayquote}

\subsection{Problématiques liées à la notation de la musique contemporaine} 
\label{subsec:pbmatiquesMusiqueContemporaine}

%L'objectif du présent stage est de créer un outil informatique permettant de noter la musique contemporaine\footnote{La musique contemporaine représente les différents courants de musique savante apparus après la fin de la Seconde Guerre mondiale} et les compositions multimédias. Aussi, il est nécessaire de résumer les problématiques rencontrées lors de la notation de telles œuvres. 

La musique contemporaine fait souvent un usage inédit des instruments de musique en proposant de nouveaux modes de jeu. 
Aussi, de nouveaux symboles sont inventés par les compositeurs pour fixer ces effets sur la portée. 
%
L'un des premiers éléments, discuté précédemment, est donc le problème de la notation des gestes du musicien. 
Dans sa pièce \textit{In lieblicher Blaue…}, Karim Haddad représente par exemple sous forme de courbes reliant des symboles rythmiques la pression, le placement et la rapidité du jeu d'archer sur une contrebasse (voir annexe~\ref{fig:exempleKarimHaddad}).

Un deuxième problème fondamental est celui de la notation de l'\textit{électronique}\footnote{Le dénomination \textit{Live Electronics} est utilisée pour qualifier l'utilisation de dispositifs électroniques dans une pièce musicale.} dans les pièces contemporaines. 
Les compositeurs ont en effet besoin, par exemple, d'\textit{écrire} les effets appliqués aux sons (réverbération, amplification, écho…). 
Dans sa pièce \textit{Tutti 157} (1965), Karlheinz Stockhausen note avec des courbes les variations des fréquences de filtrage appliquées au son, ainsi que les mouvements du potentiomètre gérant l'amplification sonore (voir annexe \ref{sec:exempleStockhausen}). Stockhausen note également dans sa pièce la variation de la distance entre les micros et la source sonore (un gong). Cela montre que le système de captation, en plus du système de diffusion, peut être pris en compte lors de la notation de dispositifs électroniques.  
%
Noter la manière dont le son est diffusé et spatialisé est également un réel enjeu de la production musicale contemporaine \cite{ellberger2015}. 
Par exemple, dans sa pièce \textit{Tak-Sîm} (2012), Alireza Farhang modélise la spatialisation du son sur les quatre enceintes de diffusion par des trajectoires en pointillés sur la partition (voir annexe \ref{sec:exempleAlirezaFarhang}).

Les nouveaux symboles inventés par les compositeurs contemporains sont en général répertoriés et expliqués au début des partitions. Les installations techniques requises par les pièces y sont aussi documentées. De fait, la notation de la musique contemporaine est indissociable d'une forme de documentation accompagnant la lecture d'une partition. L'annexe \ref{sec:schemaInstallationFluoresceComplet} présente le schéma d'installation pour la pièce \textit{Fluoresce} (2012) de Rama Gottfried. Le schéma est présenté en début de partition.

Un dernier problème soulévé par les pratiques musicales actuelles vient de la cohabitation entre notation standard (traditionnelle) et symboles \og libres \fg dans les partitions de musique mixte\footnote{Type de musique qui mélange instruments traditionnels et informatique temps-réel.}. Le difficulté vient de l'alignement des symboles entre voix superposées. En effet, dans la notation standard, la relation entre la durée d'une note et l'espace horizontal occupé dans la portée n'est pas proportionnelle, pour des raisons esthétiques ou de lisibilité. Les logiciels de gravure musicale\footnote{Les logiciels dédiés à la production d'un rendu final (imprimable) des partitions.} implémentent, la plupart du temps, un algorithme d'optimisation pour calculer l'espacement horizontal entre les notes de la portée \cite{solomon2011}.
En revanche, beaucoup de symboles permettant de noter l'électronique détiennent une relation proportionnelle au temps (courbes, nuages de points, éléments graphiques arbitraires…).
Aussi, procéder à un alignement entre une voix instrumentale et une voix électronique est parfois délicat. Dans \cite{bresson2008}, Jean Bresson et Carlos Agon choisissent d'appliquer une distorsion graphique aux éléments qui observe une relation linéaire au temps. Ces éléments sont formatés pour être alignés avec les éléments rythmiques (les notes).

\bigskip
En prenant en compte l'ensemble des challenges soulevés ci-dessus, est-il encore possible, voir même utile, de vouloir noter la musique contemporaine ?
Même si la pluralité des approches musicales amène à l'éclatement d'une pratique notationnelle unique, la partition n'en reste pas moins \og un moyen pour le compositeur de penser sa musique, la décrire et la communiquer grâce à un système de représentation symbolique \fg \cite{bresson2008}.
En réponse à la nécessité d'une pluralité notationnelle, le projet \textit{symbolist}, qui constitue le cadre du présent stage de recherche, a été impulsé par Jean Bresson et Rama Gottfried, dans l'optique d'offrir aux compositeurs multimédias et de musique contemporaine un outil informatique pour la spécification de leur propre notation.

	
	\section{Noter la musique multimédia}
	\label{sec:noterLaMusiqueMultimedia}
	Même si la pluralité des approches musicales amène à l'éclatement d'une pratique notationnelle unique, la partition n'en reste pas moins \og un moyen pour le compositeur de penser sa musique, la décrire et la communiquer grâce à un système de représentation symbolique \fg \cite{bresson2008}.
En effet, le projet \textit{symbolist} (voir section \todo{ref vers section}), qui constitue le cadre du présent stage de recherche, a été impulsé par Jean Bresson et Rama Gottfried, dans l'optique d'offrir aux compositeurs de "musique multimédia" un outil informatique pour la spécification de leur propre notation.
Par conséquent, cette partie s'intéresse à la définition de la "musique multimédia" et expose les problématiques liées à son écriture.

\subsection{Définition}

La \emph{musique multimédia}, ou \emph{composition multimédia} (\textit{media composition} en anglais), est un expression employée par Rama Gottfried dans \cite{gottfried2017}; elle désigne une pratique qui englobe la simple composition musicale.
La suivante définition d'une telle musique est proposée au lecteur : la musique multimédia décrit une pratique compositionnelle et artistique, qui, par l'entremise de moyens et supports divers empruntés au domaine audiovisuel (instruments de musique, ordinateurs, installation physique…), se constitue en vecteur d'une expérience multisensorielle (sonore, visuelle, tactile…).

La distinction est faite entre la musique multimédia et la performance artistique, en ce que la première implique une dimension sonore et musicale récurrente lors de son exécution.

Comme illustration de la musique multimédia, un fragment de la partition de la pièce \textit{Fluoresce} (R. Gottfried, 2012) est présentée en figure \ref{fig:schemaInstallationFluoresce}.

\begin{figure}[H]
	\centering
	\includegraphics[keepaspectratio=true, width=\textwidth]{Notation/i/schemaInstallationFluoresce.png}
	\caption[Schéma de branchement pour la pièce \textit{Fluoresce} par Rama Gottfried]{Schéma de branchement pour la pièce \textit{Fluoresce} par Rama Gottfried}
	\label{fig:schemaInstallationFluoresce}			
\end{figure}
\begin{center}
\small \it Cette figure présente le schéma de câblage des différents éléments qui vont servir à l'exécution de la pièce. La performance fait intervenir un joueur de violoncelle (\textit{Violoncello}); le son produit par le violoncelle est capté et envoyé à un ordinateur (\textit{Client computer}) sur lequel est lancé la station audionumérique\footnote{Une station audionumérique (acronyme DAW, de l'anglais digital audio workstation) désigne une station de travail dédiée à l'audionumérique. C'est un ensemble d'outils électroniques, conçu pour enregistrer, éditer, manipuler, créer et lire des contenus audionumériques. -- Wikipédia} \textit{Ableton Live} associé au plugin \textit{MaxForLive}\footnote{MaxForLive est un plugin permettant l'intégration du logiciel Max/MSP à la station audionumérique Ableton Live. Voir plus loin pour plus de détails sur Max/MSP.}. Ableton Live va répartir le signal audio sur les systèmes WFS et HOA (\textit{Wave Field Synthesis} et \textit{High Order Ambisonics}\footnote{\textit{Wave Field Synthesis} et \textit{High Order Ambisonics} sont deux systèmes de diffusion de sons spatialisés basés sur l'utilisation d'un grand nombre d'enceintes.}) via la liaison MADI\footnote{MADI (Multichannel Audio Digital Interface) est une liaison audionumérique définissant un protocole capable d'embarquer 64 canaux audio simultanément}. Le plugin MaxForLive va s'occuper d'envoyer des messages OSC\footnote{Open Sound Controle. Protocole décrivant des messages sous forme d'urls. \textbf{Voir le chapitre} pour plus de détails.} \texttt{/boids/1}, \texttt{/boids/2} et \texttt{/boids/3} (directives de déclenchement d'effets sonores), via UDP, aux systèmes WFS et HOA. Le schéma de branchement complet est consultable en annexe \ref{sec:schemaInstallationFluoresceComplet} page~\pageref{sec:schemaInstallationFluoresceComplet}.
\end{center}

Cette pièce met bien en place des moyens audiovisuels (un violoncelle, des ordinateurs, et deux systèmes de diffusion sonore…) pour traduire une expérience musicale; ici l'expérience est purement sonore, il n'y a pas de dimension multisensorielle. La pièce \textit{Fluoresce} est un type de musique appelée "musique mixte", c'est à dire qui mélange instruments traditionnels et informatique.

Sur le schéma, un ordinateur sur scène envoie le message OSC \texttt{/cue/time} au \textit{Client computer}. Le message \texttt{/cue/time} transmet un référentiel temporel permettant la synchronisation entre le jeu du violoncelliste et le \textit{Client computer}. 
Dans le cas de musique mixte, la question de la synchronisation humain/machine et du suivi de partition est centrale (voir la section suivante). 
En effet, même si une pièce de musique multimédia peut se détacher de toutes notions de métrique, de mélodie et d'harmonie, elle ne peut se détacher de la temporalité des évènements qui la composent et qui nécessite un interaction synchrone entre les acteurs de la pièce. 


	

%%%%% CONCLUSION %%%%%
\chapter{Conclusion}
\input{Paratextes/conclusion.tex}

\stepcounter{page}
\pagenumbering{Roman}

%%%%% BIBLIOGRAPHIE %%%%%
\printbibliography
\addcontentsline{toc}{chapter}{Bibliographie}

%%%%% TABLE DES FIGURES %%%%%
\listoffigures
\addcontentsline{toc}{chapter}{Table des figures}

%%%%% GLOSSAIRE %%%%%
\printglossary[title={Glossaire}, toctitle={Glossaire}]

%%%%% ANNEXES %%%%%
\lhead{} % remove the left part of the header (section name)
\appendix
\romanchapter{Annexes}
% Pour faire une référence d'une annexe
% (Annexe \ref{sec:nomsection} page~\pageref{sec:nomsection})

\section{Deux transcriptions de la pièce \textit{Reflets de l'ombre}, Carmine E. Cella, 2013}
\label{sec:refletsDeLOmbre}
\begin{figure}[H]
	\centering
	\includegraphics[keepaspectratio=true, width=0.8\textwidth]{Annexes/i/refletsDeLOmbreFantaisie.jpg}
	\caption{Extrait d'une première version de \textit{Reflets de l'ombre} par Carmine E. Cella}
	\medskip
	\small
	\textit{Cette première version représente la pièce en termes de variations du timbre du son.
	La musique y est notée de manière quasi morphologique, avec quelques ajouts de symboles de la gravure standard.}	
	\label{fig:refletsDeLOmbreFantaisie}
\end{figure}

\begin{figure}[H]
	\centering
	\includegraphics[keepaspectratio=true, width=0.9\textwidth]{Annexes/i/refletsDeLOmbreReel.png}
	\caption{Extrait d'une version éxecutée par un orchestre de \textit{Reflets de l'ombre} par Carmine E. Cella}
	\label{fig:refletsDeLOmbreReel}
	\medskip
	\small
	\it
	Cette version de la pièce, qui est destinée à l'orchestre exécutant, a été remaniée vis à vis de la version de la figure \ref{fig:refletsDeLOmbreFantaisie}. Cependant, les formes du son se retrouvent même dans cette transcription plus littérale.
\end{figure}
\clearpage

\section{Exemple de texte neumé}
\label{sec:exempleTexteNeume}
\begin{figure}[H]
	\centering
	\includegraphics[keepaspectratio=true, width=\textwidth]{Annexes/i/neumes.jpg}
	\caption{Exemple de texte neumé}
	\medskip
	\small	
	Source : 1774, Martin Gerbert, \textit{De cantu et musica sacra a prima Ecclesiae aetate usque ad praesens Tempus}, St. Blasien, Typis San-Blasianis, t. I, t. II. - \it Les neumes sont inscrits au-dessus des mots du chant liturgique (ici le chant Gloria de l'ordinaire de la messe). Les inflexions de la voix sont notées par des symboles assez simples, combinaisons de courbes et points essentiellement. L'idée étant seulement de donner un repère visuel au chantre (moine exécutant la liturgie), qui connaît déjà par cœur la pièce.
	\label{fig:neumes}
\end{figure}
\clearpage

\section{La notation carrée}
\label{sec:exempleNotationCarree}
\begin{figure}[H]
	\centering
	\includegraphics[keepaspectratio=true, width=0.8\textwidth]{Annexes/i/notationCarree.jpg}
	\caption{Exemple de notation carrée "imprimée"}
	\medskip
	\small
	Source : 1499, Jean Highman, \textit{Missale Leodiense}, Paris - \textit{Les notes sont représentées en notation carrée toujours au-dessus du texte chanté. L'exemple donné ci-dessus est une partition imprimée de la fin du XVème siècle, prouvant que les pratiques notationnelles perdurent même au-delà des périodes définies.}
	\label{fig:notationCarree}
\end{figure}
\clearpage

\section{La notation épurée de l'époque baroque}
\label{sec:exempleMusiqueBaroque}
\begin{figure}[H]
	\centering
	\includegraphics[keepaspectratio=true, width=\textwidth]{Annexes/i/exempleMusiqueBaroque.jpeg}
	\caption{Extrait de la pièce \textit{La Pucelle}, par François Couperin, 1690-1710}
	\medskip
	\small
	\it
	Source : Bibliothèque nationale de France, département Musique, VM7-1156.
	En 1, des symboles au-dessus des notes donnant des indications d'interprétation pour le soliste (+, vaguelettes). En 2, les numéros au-dessus des notes de la voix la plus basse (le morceau est une système à quatre voix) indiquent l'harmonie. C'est une manière de simplifier la notation des accords. 	
	\label{fig:exempleMusiqueBaroque}
\end{figure}
\clearpage

\section{La notation de l'interprétation}
\label{sec:exempleNotationInterpretation}
\begin{figure}[H]
	\centering
	\includegraphics[keepaspectratio=true, width=0.9\textwidth]{Annexes/i/exempleNotationInterpretation.png}
	\caption{Extrait de la \textit{Sonate pour piano, N.32, Opus 111}, Ludwig van Beethoven, 1821-1822}
	\label{fig:exempleNotationInterpretation}	
	\medskip
	\small
	\it
	Source : IMSLP, Petrucci Music Library. En 1, des numéros indiquant le doigté à utiliser pour la pièce.
	En 2, des indications de nuances de jeu : pp pour pianissimo, sfp pour sforzando-piano. En 3, symbole indiquant l'utilisation de la pédale du piano pour jouer la phrase. 	
\end{figure}

\section{Exemple de partition graphique du XXème siècle}
\label{sec:exempleAnestisLogothetis}
\begin{figure}[H]
	\centering
	\includegraphics[keepaspectratio=true, width=0.9\textwidth]{Annexes/i/exempleAnestisLogothetis.jpg}
	\caption{Partition de \textit{Ghia tin ora} (Pour l'heure), Anestis Logothetis, 1975} 	
	\label{fig:exempleAnestisLogothetis}
	\medskip
	\small
	\it	
	Source : \url{http://schlachten.org/artist/logothetis-ensemble/}. Dans cette partition de la deuxième moitié du XXème siècle les nuages de points et les courbes sont préférés à l'expression sur la portée. L'influence de la musique électroacoustique sur la notation musicale se traduit par une représentation proche de la morphologie du son (représentation de la forme d'ondes).
\end{figure}
\clearpage

\rotatebox{90}{
\begin{minipage}{0.95\textheight}
	\section{Notation du geste dans la musique contemporaine}
	\label{sec:exempleKarimHaddad}
	\includegraphics[keepaspectratio=true, width=\textwidth, height=\textheight]{Annexes/i/exempleKarimHaddad.png}
	\captionof{figure}{Fragment de la pièce \textit{In lieblicher Blaue…}, Karim Haddad}
	\label{fig:exempleKarimHaddad}    	
    	\medskip
	\small
	En 1, notation de la pression de l'archet sur la contrebasse. En 2, notation du placement de l'archet sur la contrebasse. En 3, notation de la vitesse de déplacement de l'archet sur la contrebasse.
\end{minipage}}
\clearpage

\rotatebox{90}{
\begin{minipage}{0.95\textheight}
	\section{Notation de la spatialisation du son dans la musique contemporaine}
	\label{sec:exempleAlirezaFarhang}
	\includegraphics[keepaspectratio=true, width=\textwidth]{Annexes/i/exempleAlirezaFarhang.png}
	\captionof{figure}{Fragment de la pièce \textit{Tak Sîm}, Alireza Farhang, 2012}
	\label{fig:exempleAlirezaFarhang}	
	\medskip
	\small
	\it
	En rouge, la description des trajectoires de déplacement du son dans le système de diffusion de la pièce.
\end{minipage}}
\clearpage

\rotatebox{90}{
\begin{minipage}{0.95\textheight}
	\section{Notation des effets appliqués au son dans la musique contemporaine}
	\label{sec:exempleStockhausen}
	\includegraphics[keepaspectratio=true, width=\textwidth]{Annexes/i/exempleStockhausen.jpg}
	\captionof{figure}{Fragment de la pièce \textit{Tutti 157}, Karlheinz Stockhausen, 1964}
	\label{fig:exempleStockhausen}	
	\medskip
	\small
	\it
	En 1, la variation de la distance entre le micro et le gong. En 2, la variation de la bande de fréquence du son par filtrage.
\end{minipage}}
\clearpage

\section{Schéma complet de branchement pour la pièce \textit{Fluoresce}, Rama Gottfried, 2012}				\label{sec:schemaInstallationFluoresceComplet}
\begin{figure}[H]
	\centering	
	\includegraphics[keepaspectratio=true, width=\textwidth]{Annexes/i/schemaInstallationFluoresceComplet.png}
	\caption{Schéma complet de branchement pour la pièce \textit{Fluoresce}, Rama Gottfried, 2012}
	\label{fig:schemaInstallationFluoresceComplet}
\end{figure}
\begin{center}
\small
\it
La performance fait intervenir un joueur de violoncelle (Violoncello); le son produit par le violoncelle est capté et envoyé à un ordinateur (Client computer) sur lequel est lancé la station audionumérique (voir section \ref{subsec:stationAudionum}) Ableton Live associé au plugin MaxForLive\footnote{MaxForLive est un plugin permettant l'intégration de Max/MSP (voir section \ref{subsec:programmationVisuelle})} à la station audionumérique Ableton Live. Ableton Live va répartir le signal audio sur les systèmes WFS et HOA (Wave Field Synthesis et High Order Ambisonics\footnote{\textit{Wave Field Synthesis} et \textit{High Order Ambisonics} sont deux systèmes de diffusion de sons spatialisés basés sur l'utilisation d'un grand nombre d'enceintes.}) via la liaison MADI\footnote{MADI (Multichannel Audio Digital Interface) est une liaison audionumérique définissant un protocole capable d'embarquer 64 canaux audio simultanément}. Le plugin MaxForLive va s'occuper d'envoyer des messages OSC (voir section \ref{subsec:interoperabilite}) \texttt{/boids/1}, \texttt{/boids/2} et \texttt{/boids/3} (directives de déclenchement d'effets sonores), via UDP, aux systèmes WFS et HOA.
\end{center}

\section{Notation de la musique dans Max avec les libraires bach et dada}
\label{sec:exempleBachDada}
\begin{figure}[H]
	\centering
	\includegraphics[keepaspectratio=true, height=0.45\textheight, width=\textwidth]{Annexes/i/exempleBachScore.png}
	\includegraphics[keepaspectratio=true, height=0.45\textheight]{Annexes/i/exempleDadaCatart.png}
	\caption{Deux objets pour la notation de la musique dans Max : bach.score et dada.catart}
	\label{fig:exempleBachDada}
	\medskip
	\small
	\it 	
	Dans la figure du haut, la page de présentation de l'objet \textbf{bach.score} qui est un éditeur de notation traditionnelle, apportant une dimension de composition assistée par ordinateur au langage Max. Dans la figure du bas, la page de présentation de l'objet \textbf{dada.catart} qui permet de représenter la musique sous forme de nuages de points. L'exploration du nuage de points avec la souris fait apparaître la transcription en notation traditionnelle sur la portée en bas à droite.
\end{figure}

\section{Fragment de la pièce \textit{Lucky Wok}, Mike Solomon}
\label{sec:luckyWokSolomon}
\begin{figure}[H]
	\centering
	\includegraphics[keepaspectratio=true, width=0.92\textwidth]{Annexes/i/lucky.png}
	\caption{Fragment de la pièce \textit{Lucky Wok}, Mike Solomon}
	\medskip
	\small
	\it
	L'ensemble des symboles et effets de la partition sont produits avec Lilypond. 	
	\label{fig:luckyWokSolomon}
\end{figure}

\section{Exemple de partition créée avec IanniX}
\label{sec:exemplePartitionIannix}
\begin{figure}[H]
	\centering
	\includegraphics[keepaspectratio=true, width=0.92\textwidth]{Annexes/i/exemplePartitionIannix.jpg}
	\caption{Exemple de partition créée avec IanniX}
	\medskip
	\small
	\it
	Les barres rouges représentent les curseurs avançant le long des courbes blanches.
	Le caractère continue des courbes et leur superposition fait penser à une description morphologique de l'onde sonore. -- Source : Blog de Nicolas Boillot, \url{https://www.fluate.net/code/tools/} 	
	\label{fig:exemplePartitionIannix}
\end{figure}

\section{Exemple de partition interactive avec i-score}
\label{sec:exempleIScore}
\begin{figure}[H]
	\centering
	\includegraphics[keepaspectratio=true, width=\textwidth]{Annexes/i/exempleIScore.jpg}
	\caption{Exemple de partition interactive avec i-score}
	\medskip
	\small
	\it  	
	\label{fig:exempleIScore}
\end{figure}

\section{Vue d'une partition \textsc{INScore} et du script associé}
\label{sec:exempleINScore}
\begin{figure}[H]
	\centering
	\includegraphics[keepaspectratio=true, width=\textwidth]{Annexes/i/exempleINScore.png}
	\begin{lstlisting}[language = python, linewidth = \textwidth]
# Suppression des précédents éléments de la scene
/ITL/scene/* del;

# Définition du titre de la scene
/ITL/scene/title set txt "This is my first score!";
/ITL/scene/title scale 3;
/ITL/scene/title y -0.6;
/ITL/scene/title fontFamily Zapfino;

# Mise en forme du cadre 
/ITL/scene/frame set rect 1.5 0.5;
/ITL/scene/frame color 230 230 230;

# Définition des notes de la portée
/ITL/scene/score set gmn '[ \meter<"4/4"> \key<-1> a f g c c g a f ]';
/ITL/scene/score scale 0.6;

# Création du curseur
/ITL/scene/cursor set ellipse 0.06 0.06;
/ITL/scene/cursor color 100 100 250;

# Synchronisation de l'objet score et cursor via l'objet sync
/ITL/scene/sync cursor score;

# Mise en mouvement du cursor le long de l'objet score
# via assignation d'une valeur de tempo
/ITL/scene/cursor tempo 60; 
	\end{lstlisting}
	\caption{Exemple de partition et de script \textsc{INScore}}
	\medskip
	\small
	\it
	La méthode \textbf{set gmn} (l.15) appelé sur l'objet \textbf{score} définit le contenu de la portée au format GUIDO. L'appel à l'objet \textbf{sync} (l.23) lie l'objet \textbf{cursor} et l'objet \textbf{score}. Ici, l'objet \textbf{cursor} devient esclave de l'objet \textbf{score}.  	
	\label{fig:exempleINScore}
\end{figure}

%\section{Vue de l'interface graphique de EAnalysis}
%\label{sec:exempleVueEAnalysis}
%\begin{figure}[H]
%	\centering
%	\includegraphics[keepaspectratio=true, width=0.92\textwidth]{Annexes/i/exempleVueEAnalysis.png}
%	\caption{Vue de l'interface graphique de EAnalysis}
%	\medskip
%	\small
%	\it
%	Plusieurs vues de la pièce \textit{Étude aux Chemins de Fer} dans l'environnement \textit{EAnalysis}.
%	De bas en haut : au premier niveau, la vue sous forme d'ondes; au deuxième niveau, le spectrogramme; au troisième niveau, la présentation des graphic et analytic events sur une ligne de temps parallèle.  	
%	\label{fig:exempleVueEAnalysis}
%\end{figure}

\end{document}
